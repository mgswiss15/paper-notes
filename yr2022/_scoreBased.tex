\clearpage

\section{Score-based models}\label{sec:scorebased}\index{score-based models}

\begin{notebox}
\textbf{Paper: } \fullcite{song_generative_2020}
\vspace{5pt}

\href{https://papers.nips.cc/paper/2019/file/3001ef257407d5a371a96dcd947c7d93-Reviews.html}{reviews}
\hspace{1cm}
\href{https://github.com/yang-song/score_sde_pytorch}{code}
\hspace{1cm}
\href{run:/home/magda/Dropbox/Zot/Song_Ermon_2020_Generative Modeling by Estimating Gradients of the Data Distribution.pdf}{Local pdf}
\vspace{3pt}

Read with Simon\index{Simon} on 9/2/2022
\hfill Notes taken: 14/2/2022 \index{February 2022}
\end{notebox}

\begin{notebox}[colback=red!5]
\tldr 
\end{notebox}

\begin{notebox}[colback=yellow!5]
\textbf{Notes:} 
\begin{itemize}[nosep]
\item 
\end{itemize}
\end{notebox}


Generative model where data are generated through Langevin dynamics \parencite{welling_bayesian_nodate} using gradients estimated through score matching \parencite{hyvarinen_estimation_nodate}.

They perturb the data with small Gaussian noise to ensure the data don't ley 