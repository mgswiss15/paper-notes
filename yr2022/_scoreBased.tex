\clearpage

\section{Score-based models}\label{sec:scorebased}\index{score-based models}

\begin{notebox}
\textbf{Paper: } \fullcite{song_generative_2020}
\vspace{5pt}

\href{https://papers.nips.cc/paper/2019/file/3001ef257407d5a371a96dcd947c7d93-Reviews.html}{reviews}
\hspace{1cm}
\href{https://github.com/yang-song/score_sde_pytorch}{code}
\hspace{1cm}
\href{run:/home/magda/Dropbox/Zot/Song_Ermon_2020_Generative Modeling by Estimating Gradients of the Data Distribution.pdf}{Local pdf}
\vspace{3pt}

Read with Simon\index{Simon} on 9/2/2022
\hfill Notes taken: 14/2/2022 \index{February 2022}
\end{notebox}

\begin{notebox}[colback=red!5]
\tldr 
\end{notebox}

\begin{notebox}[colback=yellow!5]
\textbf{Notes:} 
\begin{itemize}[nosep]
\item 
\end{itemize}
\end{notebox}


Generative model where data are generated through Langevin dynamics \parencite{welling_bayesian_nodate} using gradients estimated through score matching \parencite{hyvarinen_estimation_nodate}.

They perturb the data with small Gaussian noise to ensure the data don't lie on a low dimensional manifold within the original feature space which would make the gradients ill-defined. They also propose to anneal the Langevin dynamics sampling process with gradually decreasing noise levels.

The model is based on estimating the (Stein) score\index{index} that is the gradient of the log data density with respect to the data $\nabla_{\rvx} \log p(\rvx)$ (though standard score function in statistics is the gradient with respect to the parameters of the distribution at a fixed data point - this is acknowledged in the original [33] paper of Jordan's group but not in this paper).

The score network $s_\theta: R^D \to R^D$ shall be trained to estimate the score without estimating the data likelihood $p(\rvx)$ first. Once trained, it can be used within Langevin dynamics sampling to generate new data examples.

The score matching objective is the minimization of $0.5 \E_{p_{data}} \Vert s_{\theta}(\rvx) - \nabla_{\rvx} \log p(\rvx) \Vert_2^2$ which has been show (in \cite{hyvarinen_estimation_nodate} to be equivalent to the minimization of
\begin{align*}
\E_{p_{data}}(\rvx) \left[ tr(\nabla_\rvx s_\theta(\rvx)) + \frac{1}{2} \Vert s_\theta(\rvx) \Vert_2^2 \right]
\end{align*}

They use score matching from \parencite{hyvarinen_estimation_nodate}
